\documentclass[a4paper,12pt]{article}

%\usepackage{graphicx}
\usepackage{amssymb,amsmath}
%\numberwithin{equation}{section}
%\usepackage[shadow]{todonotes}
\usepackage{enumitem}
\usepackage{hyperref}
%\usepackage[round,comma]{natbib}
%\usepackage{showkeys}

\setlength{\topmargin}{-25mm}
\setlength{\oddsidemargin}{0mm}
\setlength{\textwidth}{160mm}
\setlength{\textheight}{240mm}

\newcommand{\ws}{w_*}
\newcommand{\w}{\omega}
\newcommand{\wmin}{\w_{\rm{min}}}
\newcommand{\xmin}{x_{\rm{min}}}
\newcommand{\xt}{x_{\rm{th}}}

\begin{document}

\title{Multi-species population balance equation}
\author{Gustav W Delius}
\maketitle

\section{Phytoplankton}
\subsection{Population balance equation}
We specialise the population balance equation \cite[eq.(2.10)]{cuesta_sheldon_2016} in the paper
to the case of a discrete set of species, i.e.,
\begin{equation}\label{dis}
p(w, \ws) = \sum_{i=1}^M \delta(\ws-w_i)\,w_i^{-\xi}p_i(w/w_i).
\end{equation}
Substituting this into eq.(2.10), using the scaling properties of
the rates
\begin{align*}
G(w,\ws)&=\ws^{1-\xi}g(w/\ws),&
K(w,\ws)&=\ws^{-\xi}k(w/\ws),\\
M(w,\ws)&=\ws^{-\xi}m(w/\ws),
&Q(w,w')&=q(w/w')/w',
\end{align*}
and performing a change of variables to $\w =w/w_i$,
we obtain the equation
\[\begin{split}\label{pbe}
w_i^\xi\partial_tp_i(\w)&=
-\partial_\w\left[g(\w)p_i(\w)\right]
+2\int q(\w/\w')\left[k(\w')p_i(\w')\right]\w'^{-1}d\w'\\
&\qquad-\left[k(\w)+m(\w)\right]p_i(\w)
\end{split}\]
for all $\w\in[\wmin,1]$, were $\wmin$ is the smallest size possible for any cell.

To see that the integral is a convolution integral we write $\w=e^x$ and $\w'=e^y$ and 
observe that then $\w'^{-1}d\w'=dy$ so that the integral takes the form
\[\begin{split}
\int q(\w/\w')\left[k(\w')p_i(\w')\right]\w'^{-1}d\w'&=\int q(e^{x-y})k(e^y)p_i(e^y)dy\\
&=\int_{\xmin}^0 f_1(x-y)g_1(y)dy=C_1(x).
\end{split}\]
The subscript $1$ is to distinguish this convolution integral from two others arising when we start
including predation in the model.
We need to calculate the convolution integral for all $x\in[\xmin,0]$, where
$\xmin=\log\wmin$ is the smallest possible cell size. 
The reason we can restrict the integral to run only from $\xt$ to $0$,
where $\xt$ is the log of the threshold size below which not division is
taking place, is that the factor $g(y)$ is zero outside that interval anyway.  
With $y$ ranging only over the interval $[\xt,0]$, the largest region in
which the function $f$ gets evaluated is the interval $[\xmin, -\xt]$. Thus we can replace
$f$ by a periodic function $\bar{f}$ with period $L=-\xmin$, without changing the value
of the integral, because $\bar{f}$ is zero on the interval from $0$ to $-xt$ and agrees with
$f$ on the interval $[\xmin, 0]$. We can then
also replace $g$ by a periodic function $\bar{g}$ with the same period $L$, which again
does not change the integral because $\bar{g}$ agrees with $g$ on the interval $[\xmin, 0]$
and is not used outside that interval. This gives us 
\[
C_1(x)=\int_{\xmin}^0\bar{f}(x-y)\bar{g}(y)dy.
\]
Because this is an integral over a complete period of the two periodic functions $\bar{f}$ and
$\bar{g}$ it can be calculated by Fast Fourier Transform. 

We also express the derivative in the birth term in terms of the variable $x=\log\w$ so that
it too can be calculated with spectral  methods: $\partial_w=w^{-1}\partial_x$.

\subsection{Nutrient}
Substituting the discrete set of species expression \eqref{dis} into the expression 
\cite[(2.13)]{cuesta_sheldon_2016} for
the rate of nutrient consumption, and choosing units for the nutrient concentration so that $\theta=1$,
we get
\begin{equation}
\sigma(N,p)=a(N)\sum_{i=1}^M w_i^{2-\xi-\gamma}\int\w^\alpha p_i(\w)d\w.
\end{equation}
Again performing the change of variable $x=\log(\w)$ in the integral gives
\[\int\w^\alpha p_i(\w)d\w=\int\w^{\alpha+1} p_i(\w)dx.\]
The reason we prefer the integral in the form on the right-hand side is that in our code
we will work with logarithmically spaced  intervals in $\w$ which translates to equally-spaced
intervals in $x$, which then simplifies the calculation of the integral with respect to $dx$ 
by the Riemann sum. For equally-spaced interpolation points and a periodic integrand the
second-order trapezoidal rule simplifies to the Riemann sum.

\section{Predation}
We now allow cells to eat other cells. We assume that the rate at which a given cell of size $w$
eats another givem cell of size $w'$ is given by $w^\nu S(w/w')$. 
This will add two additional terms to the right-hand side of the population
balance equation \eqref{pbe}: one additional death term and one additional growth term.
The death term takes the form
\[
-p_i(\w)\sum_{j=1}^M(w_i-w_j)^{\xi}\int S(\w'/\w_j)\w'^\nu p_j(\w')dw',
\]
where $\w_j=w/w_j$.


\begin{thebibliography}{9}

\bibitem{cuesta_sheldon_2016}
Cuesta, J.A., Delius, G.W., Law, R., 2016. Sheldon Spectrum and the Plankton Paradox: Two Sides of the Same Coin. A trait-based plankton size-spectrum model. arXiv preprint arXiv:1607.04158.

\end{thebibliography}


\end{document}